%%% LaTeX Template: Designer's CV
%%%
%%% Source: http://www.howtotex.com/
%%% Feel free to distribute this template, but please keep the referal to HowToTeX.com.
%%% Date: March 2012

\documentclass[a4paper,12pt,final]{memoir}

% misc
\renewcommand{\familydefault}{bch}	% font
\pagestyle{empty} % no pagenumbering
\setlength{\parindent}{0pt} % no paragraph indentation

% required packages (add your own)
\usepackage{flowfram} % column layout
\usepackage[top=1cm,left=1cm,right=1cm,bottom=1cm]{geometry} % margins
\usepackage{graphicx} % figures
\usepackage{url} % URLs
\usepackage[usenames,dvipsnames]{xcolor} % color
\usepackage{multicol} % columns env.
\setlength{\multicolsep}{0pt}
\usepackage{paralist} % compact lists
\usepackage{tikz}

%%%%%%%%%%%%%%%%%%%%%%%%%%%%%%%%%%%%%
% Create column layout
%%%%%%%%%%%%%%%%%%%%%%%%%%%%%%%%%%%%%

% define length commands
\setlength{\vcolumnsep}{\baselineskip}
\setlength{\columnsep}{\vcolumnsep}

% frame setup (flowfram package)
% left frame
% \newflowframe{0.1\textwidth}{\textheight}{0pt}{0pt}[left]
\newlength{\LeftMainSep}
\setlength{\LeftMainSep}{2em}
% \addtolength{\LeftMainSep}{2\columnsep}
% % right frame
\newflowframe{0.9\textwidth}{\textheight}{\LeftMainSep}{0pt}[main01]

% horizontal rule between frames (using TikZ)
\renewcommand{\ffvrule}[3]{%
  \hfill
  \tikz{%
    \draw[loosely dotted,color=RoyalBlue,line width=1.5pt,yshift=-#1]
    (0,0) -- (0pt,#3);}%
  \hfill\mbox{}}
%\insertvrule{flow}{1}{flow}{2}

%%%%%%%%%%%%%%%%%%%%%%%%%%%%%%%%%%%%%
% define macros (for convience)
%%%%%%%%%%%%%%%%%%%%%%%%%%%%%%%%%%%%%
\newcommand{\Sep}{\vspace{1.5em}}
\newcommand{\SmallSep}{\vspace{0.5em}}

\newcommand{\CVSection}[1]
{\Large\textbf{#1}\par
  \SmallSep\normalsize\normalfont}

\newcommand{\CVItem}[1]
{\textbf{\color{RoyalBlue} #1}}

\newcommand{\CPP}{C\hspace{-2pt}+\hspace{-4pt}+}
%%%%%%%%%%%%%%%%%%%%%%%%%%%%%%%%%%%%%
% Begin document
%%%%%%%%%%%%%%%%%%%%%%%%%%%%%%%%%%%%%
\begin{document}


% Right frame
%%%%%%%%%%%%%%%%%%%%

\Huge\bfseries {\color{RoyalBlue}Sean McLaughlin}\\
\large\bfseries {\color{RoyalBlue}Software Engineer}
\normalsize\normalfont

\begin{flushright}\small
  \vspace{-3em}
  \begin{tabular}{l}
    \url{seanmcl@gmail.com}\\
    \url{https://github.com/seanmcl}\\
    \url{https://www.linkedin.com/in/mclaughlins}\\
    (718) 483-0902
  \end{tabular}
\end{flushright}\normalsize

\normalsize\normalfont

\vspace{-1em}

% Summary
\CVSection{Summary}
I love designing and implementing real software systems.  It is the
confluence of beautiful abstractions with real world applications that
I find the most challenging and rewarding in my work.  For example, I
am partial to mechanized correctness proofs, first-class functions,
immutable data, and rich type systems.  But I also love building a
Mesos cluster on AWS from scratch to analyze a bunch of data, and
debugging low level network problems when things go wrong.
I am seeking a position where I can use my enthusiasm,
knowledge, and experience at all levels of the software stack to solve
hard real-world problems.

% Experience
\CVSection{Work Experience}
\CVItem{April 2014 - Present, Engineers Gate LP}\\
Systems engineer.  Engineers Gate is a startup hedge fund (7
engineers) specializing in medium and high frequency electronic
trading.  My team handles market data and research infrastructure.  My
duties include provisioning and managing our production EC2 cluster,
containerizing, deploying, securing and monitoring over 40
services, and writing the bulk of our web dashboards. I also manage a small
team of contractors who help with system and network administration.
(Scala, Python, Java, Javascript, HTML, CSS, Scala.js, Docker, Ansible, AWS,
Mesos, Marathon, Chronos)
\SmallSep

\CVItem{October 2012 - April 2014, Google}\\
Software engineer, Google Maps.  I worked on the machine learning
team responsible for clustering local businesses with data
from disparate sources.  I specialized in creating data and visualization
pipelines using Flume, a stream-based functional programming library
for specifying mapreduce pipelines.  (\CPP, Python)
\SmallSep

\CVItem{August 2010 - October 2012, Jane Street Capital}\\
Software engineer, quantitative research group.
I built the in-house code review tools, OCaml compiler frontend, and
worked on a distributed real-time order marking system using the
OCaml Async futures library.  (OCaml)
\SmallSep

\CVItem{Summer 2009, National Science Foundation} \\
Taught courses on functional programming and theorem proving, Hanoi,
Vietnam. (OCaml, HOL Light)
\SmallSep

\CVItem{Summer 2007, Microsoft/INRIA} \\
I worked on the formalization of Galois Theory in the Coq proof
assistant. (OCaml, Coq)
\SmallSep

\CVItem{Summer 2004, Intel} \\
I worked on theorem-prover-based static analysis in the internal
reFLect programming language, used by Intel to verify hardware
designs. (CVC, reFLect)
\SmallSep

\CVItem{Summer 2002, IBM} \\
I wrote a compiler for an XML-based database query language. (Java)
\SmallSep

\CVItem{June 2001 - July 2002, Eventmonitor, Inc.} \\
Software engineer at a financial software startup. (Java)
\SmallSep

\CVSection{Selected Publications}

\CVItem{The Dodecahedral Conjecture} \\
Solves a problem in discrete geometry originally posed in 1943.\\
Journal of the American Mathematical Society, 2010
\SmallSep

\CVItem{Efficient Intuitionistic Theorem Proving with the Polarized Inverse Method}\\
Conference on Automated Deduction, 2009
\SmallSep

\CVItem{Imogen: Focusing the Polarized Inverse Method for Intuitionistic Propositional Logic}\\
Logic for Programming, Artificial Intelligence and Reasoning, 2008
\SmallSep

\CVItem{An interpretation of Isabelle/HOL in HOL Light}\\
International Joint Conference on Automated Reasoning, 2006
\SmallSep

\CVItem{A Proof Producing Decision Procedure for Real Arithmetic}\\
Conference on Automated Deduction, 2005
\SmallSep

\CVSection{Open Source Projects}
\CVItem{Imogen} \url{https://github.com/seanmcl/imogen}\\
A theorem prover for intuitionistic logics (Standard ML, Haskell)

\SmallSep

% Education
\CVSection{Education}
\CVItem{2004 -, Carnegie Mellon University}\\
Ph.D. candidate in Computer Science.  (Currently ABD)
\SmallSep

\CVItem{2002-2004 - New York University}\\
Masters in Computer Science
\SmallSep

\CVItem{1995-2000 - University of Michigan}\\
Bachelor of Science, Mathematics \\
Bachelor of Musical Arts, Clarinet performance \\
\SmallSep

\CVItem{1994-1995 - Interlochen Arts Academy}\\

\CVSection{Awards}
\CVItem{2000 - AMS, MAA, SIAM Morgan Prize for Outstanding Mathematics Research}
\SmallSep

\CVSection{Teaching experience}

\CVItem{Carnegie Mellon University}\\
Teaching Assistant in functional programming and constructive logic.
\SmallSep

\CVItem{Courant Institute, New York University}\\
Teaching Assistant in algorithms, programming languages, and
artificial intelligence.
\SmallSep

\CVSection{Languages}

\CVItem{Used daily: }
Scala, Java, Python, JavaScript, HTML, CSS, Bash, Emacs Lisp
\SmallSep

\CVItem{Favorite:} OCaml, Standard ML, Scala, Haskell
\SmallSep

\CVItem{Used previously:} \CPP
\SmallSep

\CVSection{Favorite Tools}

\CVItem{Systems: }
Zookeeper, Mesos, Marathon, Chronos, Docker, Ansible, ELK, Graphite,
Sensu, RPM, Nginx, HAProxy
\SmallSep

\CVItem{Libraries: }
Scala.js, ReactiveX, D3, Bootstrap, React, Flume
\SmallSep

\CVItem{AWS: }
EC2, S3, VPC, Route53, IAM
\SmallSep

\end{document}
