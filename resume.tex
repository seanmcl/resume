%%% LaTeX Template: Designer's CV
%%%
%%% Source: http://www.howtotex.com/
%%% Feel free to distribute this template, but please keep the referal to HowToTeX.com.
%%% Date: March 2012

\documentclass[a4paper,12pt,final]{memoir}

% misc
\renewcommand{\familydefault}{bch}	% font
\pagestyle{empty} % no pagenumbering
\setlength{\parindent}{0pt} % no paragraph indentation

% required packages (add your own)
\usepackage{flowfram} % column layout
\usepackage[top=1cm,left=1cm,right=1cm,bottom=1cm]{geometry} % margins
\usepackage{graphicx} % figures
\usepackage{url} % URLs
\usepackage[usenames,dvipsnames]{xcolor} % color
\usepackage{multicol} % columns env.
\setlength{\multicolsep}{0pt}
\usepackage{paralist} % compact lists
\usepackage{tikz}

%%%%%%%%%%%%%%%%%%%%%%%%%%%%%%%%%%%%%
% Create column layout
%%%%%%%%%%%%%%%%%%%%%%%%%%%%%%%%%%%%%

% define length commands
\setlength{\vcolumnsep}{\baselineskip}
\setlength{\columnsep}{\vcolumnsep}

% frame setup (flowfram package)
% left frame
% \newflowframe{0.1\textwidth}{\textheight}{0pt}{0pt}[left]
\newlength{\LeftMainSep}
\setlength{\LeftMainSep}{2em}
% \addtolength{\LeftMainSep}{2\columnsep}
% % right frame
\newflowframe{0.9\textwidth}{\textheight}{\LeftMainSep}{0pt}[main01]

% horizontal rule between frames (using TikZ)
\renewcommand{\ffvrule}[3]{%
  \hfill
  \tikz{%
    \draw[loosely dotted,color=RoyalBlue,line width=1.5pt,yshift=-#1]
    (0,0) -- (0pt,#3);}%
  \hfill\mbox{}}
%\insertvrule{flow}{1}{flow}{2}

%%%%%%%%%%%%%%%%%%%%%%%%%%%%%%%%%%%%%
% define macros (for convience)
%%%%%%%%%%%%%%%%%%%%%%%%%%%%%%%%%%%%%
\newcommand{\Sep}{\vspace{1.5em}}
\newcommand{\SmallSep}{\vspace{0.5em}}

\newcommand{\CVSection}[1]
{\Large\textbf{#1}\par
  \SmallSep\normalsize\normalfont}

\newcommand{\CVItem}[1]
{\textbf{\color{RoyalBlue} #1}}

\newcommand{\CPP}{C\hspace{-2pt}+\hspace{-4pt}+}
%%%%%%%%%%%%%%%%%%%%%%%%%%%%%%%%%%%%%
% Begin document
%%%%%%%%%%%%%%%%%%%%%%%%%%%%%%%%%%%%%
\begin{document}


% Right frame
%%%%%%%%%%%%%%%%%%%%

\Huge\bfseries {\color{RoyalBlue} Sean McLaughlin}
\normalsize\normalfont

\begin{flushright}\small
  \vspace{-3em}
  \begin{tabular}{l}
    \url{seanmcl@gmail.com}\\
    \url{https://github.com/seanmcl}\\
    (718) 483-0902
  \end{tabular}
\end{flushright}\normalsize

\normalsize\normalfont

\vspace{-1em}

% Experience
\CVSection{Work Experience}
\CVItem{October 2012 - present, Google}\\
Software engineer for Google Maps.  I work on the machine learning
team responsible for clustering local businesses with data
from disperate sources.
\SmallSep

\CVItem{August 2010 - October 2012, Jane Street Capital}\\
Quantitative research and technology group.  I worked on a
distributed real-time order marking system.
\SmallSep

\CVItem{Summer 2009, National Science Foundation} \\
Taught courses on functional programming and theorem proving, Hanoi, Vietnam.
\SmallSep

\CVItem{Summer 2007, Microsoft/INRIA} \\
I worked on the formalization of Galois Theory in the Coq proof
assistant.
\SmallSep

\CVItem{Summer 2004, Intel} \\
I worked on theorem-prover-based static analysis in the internal
reFLect programming language, used by Intel to verify hardware
designs.
\SmallSep

\CVItem{Summer 2002, IBM} \\
I wrote a compiler for an XML-based database query language.
\SmallSep

\CVItem{June 2001 - July 2002, Eventmonitor, Inc.} \\
Java programmer at a financial software startup.
\SmallSep

\CVSection{Selected Publications}

\CVItem{The Dodecahedral Conjecture} \\
Solves a problem in discrete geometry originally posed in 1943.\\
Journal of the American Mathematical Society, 2010
\SmallSep

\CVItem{Efficient Intuitionistic Theorem Proving with the Polarized Inverse Method}\\
Conference on Automated Deduction, 2009
\SmallSep

\CVItem{Imogen: Focusing the Polarized Inverse Method for Intuitionistic Propositional Logic}\\
Logic for Programming, Artificial Intelligence and Reasoning, 2008
\SmallSep

\CVItem{An interpretation of Isabelle/HOL in HOL Light}\\
International Joint Conference on Automated Reasoning, 2006
\SmallSep

\CVItem{A Proof Producing Decision Procedure for Real Arithmetic}\\
Conference on Automated Deduction, 2005
\SmallSep

\CVSection{Open Source Projects}
\CVItem{omake-mode}\\
An Emacs interface to the OCaml compiler (\url{https://github.com/seanmcl/omake-mode})
\SmallSep

\CVItem{Imogen}\\
A theorem prover for intuitionistic logics (\url{https://github.com/seanmcl/imogen})
\SmallSep

% Education
\CVSection{Education}
\CVItem{2004 - present, Carnegie Mellon University}\\
Ph.D. in Computer Science
\SmallSep

\CVItem{2002-2004 - New York University}\\
Masters in Computer Science
\SmallSep

\CVItem{1995-2000 - University of Michigan}\\
Bachelor of Science, Mathematics \\
Bachelor of Musical Arts, Clarinet performance \\
\SmallSep

\CVItem{1994-1995 - Interlochen Arts Academy}\\

% Awards
\CVSection{Awards}
\CVItem{2000 - AMS, MAA, SIAM Morgan Prize for Outstanding Mathematics Research}
\SmallSep

\CVSection{Teaching experience}

\CVItem{Carnegie Mellon University}\\
Teaching Assistant in functional programming and constructive logic.
\SmallSep

\CVItem{Courant Institute, New York University}\\
Teaching Assistant in algorithms, programming languages, and
artificial intelligence.
\SmallSep

% Skills
\CVSection{Languages}

\CVItem{Used daily}\\
 \CPP, Python, Bash, Emacs Lisp
\SmallSep

\CVItem{Used regularly}\\
OCaml, Standard ML, Haskell
\SmallSep

\CVItem{Passing knowledge}\\
Javascript, HTML, CSS, Prolog
\SmallSep

\end{document}
